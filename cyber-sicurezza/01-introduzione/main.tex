\documentclass{beamer}
\usepackage[italian]{babel}
\usepackage[utf8]{inputenc}
\usepackage[T1]{fontenc}

% Personal commands
\newcommand{\code}[1]{\mbox{\texttt{#1}}}
\newcommand{\command}[1]{\mbox{\texttt{#1}}}
\newcommand{\file}[1]{\mbox{\texttt{#1}}}

% Beamer options
\beamertemplatenavigationsymbolsempty
\setbeamertemplate{bibliography item}{}
\setbeamertemplate{caption}{\insertcaption}
\setbeamertemplate{footline}[frame number]

\title{Introduzione alla cyber security, ethical hacking e CTF}
\author[leot]{Leonardo Taccari \\ {\footnotesize \texttt{<s1069964@studenti.univpm.it>}}}
\date{}

\begin{document}

% Title of the presentation
\begin{frame}
\maketitle
\end{frame}

% Outline
\begin{frame}{Sommario}
\tableofcontents
\end{frame}

\section*{Un'occhiata a notizie recenti riguardante la cyber security}
\begin{frame}{\insertsection}
\begin{itemize}
\item Perché la cyber security è importante?
\item Proviamo a "sfogliare" i giornali delle ultime settimane
\item Come questi incidenti possono coinvolgerci?
\end{itemize}
\end{frame}

\subsection*{Ragazzo viola registro elettronico per cambiare i suoi voti}
\begin{frame}{\insertsection}{\insertsubsection}
\begin{figure}
\includegraphics[width=0.95\textwidth]{imgs/news-wired-hacker-15enne.png}
\caption{Cosa sappiamo dell'hacker 15enne che ha violato il registro elettronico
per cambiare i suoi voti, Chiara Crescenzi, Wired, 20/01/2025
\href{https://www.wired.it/article/hacker-15enne-voti-scolastici/}{(link)}}
\end{figure}
\end{frame}

\subsection*{Rubati dati personali di più di 5.5 milioni di utenti InfoCert}
\begin{frame}{\insertsection}{\insertsubsection}
\begin{figure}
\includegraphics[width=0.6\textwidth]{imgs/news-ilsole24ore-infocert.png}
\caption{Hacker contro fornitore esterno Infocert: rubati dati personali degli
utenti. La società: «Dati di Spid, firma e Pec non compromessi», Il Sole 24 Ore,
29/12/2024
\href{https://www.ilsole24ore.com/art/spid-hacker-contro-infocert-rubate-informazioni-milioni-utenti-AGnDOQ2B}{(link)}}
\end{figure}
\end{frame}

\subsection*{Fuga di dati da Volkswagen: accessibili in chiaro posizioni geografiche e dati personali dei possessori di 800000 automobili}
\begin{frame}{\insertsection}{\insertsubsection}
\begin{figure}
\includegraphics[width=0.55\textwidth]{imgs/news-spiegel-vw-dataleak.png}
\caption{Massive Data Breach at VW Raises Questions about Vehicle Privacy,
Patrick Beuth, Flüpke, Max Hoppenstedt, Michael Kreil, Marcel Rosenbach e Rina
Wilkin, Der Spiegel, 03/01/2025
\href{https://www.spiegel.de/international/business/we-know-where-you-parked-massive-data-breach-at-vw-raises-questions-about-vehicle-privacy-a-4b1cb926-2edb-42ea-92fb-5000cd378fc5}{(link)}}
\end{figure}
\end{frame}

\section{Concetti fondamentali}
\begin{frame}{\insertsection}
\end{frame}

\subsection{Cyber Security}
\begin{frame}{\insertsubsection: definizioni}
Alcune definizioni dal NIST~\footnote{National Institute of Standards
and Technology} Computer Security Resource Center (CSRC) Glossary (via
NIST SP 800-30).
\end{frame}

\subsubsection*{Cyberspace}
\begin{frame}{\insertsubsection: definizioni}{\insertsubsubsection}
\begin{block}{\insertsubsubsection}
A global domain within the information environment consisting of the
interdependent network of information systems infrastructures including
the Internet, telecommunications networks, computer systems, and
embedded processors and controllers.
\end{block}
\end{frame}

\subsubsection*{Cyber Attack}
\begin{frame}{\insertsubsection: definizioni}{\insertsubsubsection}
\begin{block}{\insertsubsubsection}
An \alert{attack}, via \alert{cyberspace}, targeting an enterprise's
use of cyberspace for the purpose of disrupting, disabling, destroying,
or maliciously controlling a computing environment/infrastructure; or
destroying the integrity of the data or stealing controlled
information.
\end{block}
\end{frame}

\subsubsection*{Cyber Security}
\begin{frame}{\insertsubsection: definizioni}{\insertsubsubsection}
\begin{block}{\insertsubsubsection}
The ability to protect or defend the use of \alert{cyberspace} from
\alert{cyber attacks}.
\end{block}
\end{frame}

\subsection{CIA: Confidentiality, Integrity, Availability}
\begin{frame}{\insertsubsection}
I pilastri della \alert{cyber security} sono costituiti dalla "triade" \alert{CIA}:
\alert{Confidentiality}, \alert{Integrity}, \alert{Availability}.
\end{frame}

\subsubsection*{Confidentiality (Confidenzialità, Riservatezza)}
\begin{frame}{\insertsubsection}{\insertsubsubsection}
\begin{block}{\insertsubsubsection}
La \alert{confidentiality} (confidenzialità, riservatezza) è la
proprietà che garantisce che le risorse sono accessibili solo ai
soggetti autorizzati.
\end{block}
\end{frame}

\subsubsection*{Integrity (Integrità)}
\begin{frame}{\insertsubsection}{\insertsubsubsection}
\begin{block}{\insertsubsubsection}
La \alert{integrity} (integrità) è la proprietà che garantisce che le
risorse non siano alterate o distrutte da soggetti non autorizzati ad
accederle.
\end{block}
\end{frame}

\subsubsection*{Availability (Disponibilità)}
\begin{frame}{\insertsubsection}{\insertsubsubsection}
\begin{block}{\insertsubsubsection}
La \alert{availability} (disponibilità) è la proprietà che garantisce un accesso
affidabile e tempestivo alle risorse da parte dei soggetti autorizzati.
\end{block}
\end{frame}

\subsubsection*{Cyber security e CIA (Confidentiality, Integrity, Availability)}
\begin{frame}{\insertsubsection}{\insertsubsubsection}
\begin{block}{\insertsubsubsection}
Quando una o più di queste proprietà viene violata si ha un problema di
\alert{cybersecurity}.
\end{block}
\end{frame}

\subsubsection*{Vulnerabilità}
\begin{frame}{\insertsubsection}{\insertsubsubsection}
\begin{block}{\insertsubsubsection}
Bug, difetto, debolezza o esposizione accidentale di un'applicazione, sistema,
dispositivo o servizio che può portare alla violazione di
\alert{confidentiality}, \alert{integrity} o \alert{availability}.
\end{block}
\end{frame}

\subsubsection*{Esempio: Registro elettronico}
\begin{frame}[allowframebreaks]{\insertsubsection}{\insertsubsubsection}
Le proprietà CIA dipendono dal sistema considerato. Cerchiamo di
contestualizzarle nel Registro elettronico!
\begin{description}
\item[confidentiality] uno studente può visualizzare solo i propri voti \\
(se uno studente potesse vedere i voti di altri studenti si avrebbe una
\alert{fuga di informazioni} (\alert{unauthorized disclosure}))
\item[integrity] uno studente può accedere ai voti solo in lettura, non
può modificarli; un docente può assegnare una votazione che va da 1 a 10 \\
(se uno studente potesse modificare i voti o un docente potesse mettere una
votazione di 11 si avrebbe una \alert{modifica non autorizzata}
(\alert{unauthorized modification}) o \alert{modifica impropria}
(\alert{improper modification}))
\item[availability] uno studente deve poter leggere i voti, un docente deve
poter assegnare i voti (se un docente non potesse assegnare delle valutazioni si
avrebbe una \alert{"ritenuta" non autorizzata} (\alert{unauthorized withholding}))
\end{description}
\end{frame}

\subsection{Ethical Hacking}
\begin{frame}{\insertsubsection}
\end{frame}

\subsection{Software Libero}
\begin{frame}{\insertsubsection}
\end{frame}

\section{Capture The Flag (CTF)}
\begin{frame}{\insertsection}
\end{frame}

\section{Conclusioni}
\begin{frame}{\insertsection}
\end{frame}

\end{document}
