\documentclass{beamer}
\usepackage[italian]{babel}
\usepackage[utf8]{inputenc}
\usepackage[T1]{fontenc}

% Personal commands
\newcommand{\code}[1]{\mbox{\texttt{#1}}}
\newcommand{\command}[1]{\mbox{\texttt{#1}}}
\newcommand{\file}[1]{\mbox{\texttt{#1}}}

% Beamer options
\beamertemplatenavigationsymbolsempty
\setbeamertemplate{bibliography item}{}
\setbeamertemplate{caption}{\insertcaption}
\setbeamertemplate{footline}[frame number]

\title{Shell e Python}
\author[leot]{Leonardo Taccari \\ {\footnotesize \texttt{<s1069964@studenti.univpm.it>}}}
\date{}

\begin{document}

% Title of the presentation
\begin{frame}
\maketitle
\end{frame}

% Outline
\begin{frame}{Sommario}
\tableofcontents
\end{frame}

\section{Shell}
\begin{frame}{\insertsection}
\end{frame}

\subsection*{Perché?}
\begin{frame}{\insertsubsection}
\begin{itemize}
\item Molto spesso, oltre all'interfaccia grafica, utilizzeremo
direttamente la linea di comando o \alert{shell}
\item Molti sistemi - per utilizzare diverse funzionalità - utilizzano la shell
\item Diversi tipi di \alert{vulnerabilità} possono darci accesso alla shell
\item La \alert{shell} è anche programmabile (è in tutto e per tutto un
linguaggio di programmazione!): possiamo scrivere degli
\alert{shell script}
\end{itemize}
\end{frame}

\subsection*{Sintassi dei comandi}
\begin{frame}[fragile]{\insertsubsection}
\begin{block}{Sintassi di un comando}
\begin{verbatim}
comando argomento_1 argomento_2 ... argomento_n
\end{verbatim}
\end{block}
\end{frame}

\subsection*{\file{stdin}, \file{stdout}, \file{stderr}}
\begin{frame}{\insertsubsection}
Ogni programma all'esecuzione ha accesso a tre \alert{file}:
\begin{description}
\item[Standard Input (stdin) ($0$)] file usato per l'input, di default la
tastiera
\item[Standard Output (stdout) ($1$)] file usato per l'output, di default lo
schermo
\item[Standard Error (stderr) ($2$)] file usato per l'output degli errori, di
default lo schermo
\end{description}
In Unix~\footnote{GNU/Linux, insieme ai sistemi BSD e macOS fa parte
dei sistemi operativi detti Unix-like} solitamente un programma fa un
singolo compito semplice. Componendo più programmi insieme è possibile
effettuare dei compiti complessi.
\end{frame}

\subsection*{Comandi base}
\begin{frame}[allowframebreaks]{\insertsubsection}
\begin{description}
\item[man] mostra le pagine di manuale, esempio: \command{man ls}
\item[cat] concatena e stampa file, esempio: \command{cat /etc/passwd}
\item[file] determina il tipo di file, esempio: \command{file flag.png}
\item[cd] cambia la directory (cartella) corrente, esempio: \command{cd cyberchallenge}
\item[ls] lista i file/directory, esempio: \command{ls /home}
\item[mkdir] crea una directory, esempio: \command{mkdir mydir}
\item[cp] copia file/directory, esempio: \command{cp orig new}
\item[mv] rinomina/muove file/directory, esempio: \command{mv src dst}
\item[rm] rimuove file/directory~\footnote{Non esiste nessun cestino! Il file viene
rimosso direttamente!}, esempio: \command{rm delete-me}
\item[head] mostra le prime n linee (o byte), esempio: \command{head -5 README.txt}
\item[tail] mostra le ultime n linee (o byte), esempio: \command{tail -5 README.txt}
\item[less] paginatore (mostra il contenuto di un file in maniera interattiva), esempio: \command{less README.txt}
\item[hexdump] mostra il "dump" di un file in esadecimale, esempio: \command{hexdump -C README.txt}
\end{description}
\end{frame}

\section{Python}
\begin{frame}{\insertsection}
\end{frame}

\subsection*{Perché?}
\begin{frame}{\insertsubsection}
\begin{itemize}
\item Semplice da utilizzare
\item Molto espressivo
\item Moltissime librerie già disponibili
\end{itemize}
\end{frame}

\subsection*{Un'occhiata al Python}
\begin{frame}{\insertsubsection}
\begin{itemize}
\item aiuto interattivo
\item variabili e tipi
\item strutture dati
\item condizioni
\item cicli
\item funzioni
\item test
\item introspezione
\end{itemize}
\end{frame}

\section{Conclusioni}
\begin{frame}{\insertsection}
\begin{itemize}
\item Abbiamo visto alcuni concetti e comandi della \alert{shell}
\item Abbiamo rispolverato il \alert{Python} in maniera interattiva
\item Durante le prossime lezioni utilizzeremo sia la shell che il Python e li
approfondiremo all'occorrenza!
\end{itemize}
\end{frame}

% FIXME: Use bibtex for that!
\section{Riferimenti}
\begin{frame}{\insertsection}
\begin{itemize}
\item \href{https://training.olicyber.it/training}{Materiale Didattico del Portale di allenamento delle Olimpiadi Italiane di Cybersicurezza} 
\item \href{https://pwn.college/}{pwn.college}
\item \href{https://missing.csail.mit.edu/}{The Missing Semester of Your CS Education}
\item \href{https://docs.python.org/3/tutorial/index.html}{The Python Tutorial}
\item \href{https://overthewire.org/wargames/bandit/}{OverTheWire Bandit}
\end{itemize}
\end{frame}

\end{document}
