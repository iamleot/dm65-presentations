\documentclass{beamer}
\usepackage[italian]{babel}
\usepackage[utf8]{inputenc}
\usepackage[T1]{fontenc}

% Personal commands
\newcommand{\code}[1]{\mbox{\texttt{#1}}}
\newcommand{\command}[1]{\mbox{\texttt{#1}}}
\newcommand{\file}[1]{\mbox{\texttt{#1}}}

% Beamer options
\beamertemplatenavigationsymbolsempty
\setbeamertemplate{bibliography item}{}
\setbeamertemplate{caption}{\insertcaption}
\setbeamertemplate{footline}[frame number]

\title{Web Security: curl e requests}
\author[leot]{Leonardo Taccari \\ {\footnotesize \texttt{<s1069964@studenti.univpm.it>}}}
\date{}

\begin{document}

% Title of the presentation
\begin{frame}
\maketitle
\end{frame}

% Outline
\begin{frame}{Sommario}
\tableofcontents
\end{frame}

\section{Un po' di pratica!}
\subsection*{curl: richiesta HTTP GET}
\begin{frame}[fragile]{\insertsection}{\insertsubsection}
\begin{itemize}
\item \alert{curl} è un client \alert{HTTP}
\item Vediamo una semplice richiesta HTTP GET!
\item Per le prove seguenti utilizzeremo \href{https://httpbin.org}{httpbin.org}
che è un server HTTP che supporta molti \alert{metodi HTTP} e ci stampa la
richiesta che abbiamo fatto formattata in JSON.
\end{itemize}
{\tiny
\begin{verbatim}
$ curl http://httpbin.org/get
{
  "args": {},
  "headers": {
    "Accept": "*/*",
    "Host": "httpbin.org",
    "User-Agent": "curl/8.12.0",
    "X-Amzn-Trace-Id": "Root=1-67ac5b9f-702782b106ebb1af7a6bfcf5"
  },
  "origin": "5.77.65.203",
  "url": "http://httpbin.org/get"
}
\end{verbatim}
}
\end{frame}

\subsection*{JSON (JavaScript Object Notation)}
\begin{frame}{\insertsection}{\insertsubsection}
\begin{itemize}
\item \alert{JSON (JavaScript Object Notation)} è un formato utilizzato per lo
scambio di dati
\item Deriva dal linguaggio di programmazione JavaScript
\item È utilizzato in moltissimi servizi Web per scambiare dati
\item Il suo tipo MIME~\footnote{Che ritroveremo negli header HTTP
\code{Accept:} per le richieste HTTP
e \code{Content-Type:} per le risposte HTTP.} è \code{application/json}
\end{itemize}
\end{frame}

\subsection*{requests: richiesta HTTP GET}
\begin{frame}[fragile,allowframebreaks]{\insertsection}{\insertsubsection}
\begin{itemize}
\item \alert{Requests} è una libreria HTTP Python
\item Vediamo la stessa richiesta utilizzando requests!
\end{itemize}
{\tiny
\begin{verbatim}
$ python3
Python 3.12.9 (main, Feb 10 2025, 00:16:27) [GCC 12.4.0] on netbsd10
Type "help", "copyright", "credits" or "license" for more information.
>>> import requests
>>> response = requests.get("http://httpbin.org/get")
>>> # response conterrà la risposta HTTP ed altre meta-informazioni
>>> # response.text è il body della risposta HTTP decodificato in str
>>> print(response.text)  
{
  "args": {},
  "headers": {
    "Accept": "*/*",
    "Accept-Encoding": "gzip, deflate, br, zstd",
    "Host": "httpbin.org",
    "User-Agent": "python-requests/2.32.3",
    "X-Amzn-Trace-Id": "Root=1-67ac5f26-5526d7f1254e39cc6c30b253"
  },
  "origin": "5.77.65.203",
  "url": "http://httpbin.org/get"
}

\end{verbatim}
}
\begin{itemize}
\item Curiosiamo altri campi della \code{response}!
\end{itemize}
{\tiny
\begin{verbatim}
>>> response.status_code  # contiene lo status code della risposta HTTP
200
>>> response.reason  # contiene lo status text della risposta HTTP
'OK'
>>> response.headers  # dict che contiene tutti gli HTTP header della response
{'Date': 'Wed, 12 Feb 2025 08:48:13 GMT', 'Content-Type': 'application/json', ... }
\end{verbatim}
}
\begin{itemize}
\item Diamo un'occhiata alla documentazione ufficiale ed al quickstart in
\url{https://requests.readthedocs.io/}
\end{itemize}
\end{frame}

\subsection*{curl: richiesta HTTP GET con parametri}
\begin{frame}[fragile,allowframebreaks]{\insertsection}{\insertsubsection}
\begin{itemize}
\item Possiamo passare parametri alle richieste HTTP in modo da ottenere una
particolare versione della risorsa
\item I parametri vengono passati tramite la \alert{query}
\item La query è preceduta da un \code{?} e contiene una coppia di
chiave-valori
\end{itemize}
{\tiny
\begin{verbatim}
$ curl http://httpbin.org/get?chiave=valore
{
  "args": {
    "chiave": "valore"
  },
  "headers": {
    ...
  },
  ...
  "url": "http://httpbin.org/get?chiave=valore"
}
\end{verbatim}
}
\begin{itemize}
\item Se vogliamo passare più chiavi-valori possiamo separiamo le chiavi con il
carattere \code{\&}, ad esempio
\small{\verb|curl 'http://httpbin.org/get?chiave1=foo&chiave2=bar|}~\footnote{Abbiamo
messo l'URL tra apici singoli \code{'} così che caratteri come
\code{\&} non vengono interpretati dalla shell... nel dubbio
mettete sempre ciò tra apici singoli!}
\item In alternativa possiamo utilizzare le opzioni \verb|--get| e \verb|--data|
di \command{curl}, in modo da non dover preoccuparci di preparare la query
string noi nell'URL.
\item Dobbiamo specificare \verb|--get| per specificare che vogliamo una
richiesta HTTP con il verbo GET perché quando passiamo \verb|--data|
curl assume che vogliamo usare il verbo POST altrimenti.
\end{itemize}
{\tiny
\begin{verbatim}
$ curl --get --data 'chiave=' http://httpbin.org/get
{
  "args": {
    "chiave": "valore"
  },
  "headers": {
    ...
  },
  ...,
  "url": "http://httpbin.org/get?chiave=valore"
}
\end{verbatim}
}
\begin{itemize}
\item Come facciamo se vogliamo inserire un \code{=} o \code{?} o altri
caratteri speciali nella query?
\item Utilizziamo l'\alert{URL encoding}!
\item In curl, invece di usare \verb|--data| basta che utilizziamo
\verb|--data-urlencode|:
\end{itemize}
{\tiny
\begin{verbatim}
$ curl --get --data-urlencode 'chiave=valore con spazi e == e ?' http://httpbin.org/get
{
  "args": {
    "chiave": "valore con spazi e == e ?"
  },
  "headers": {
    ...
  },
  ...,
  "url": "http://httpbin.org/get?chiave=valore+con+spazi+e+%3d%3d+e+%3f"
}
\end{verbatim}
}
\end{frame}

\subsection*{requests: richiesta HTTP GET con parametri}
\begin{frame}[fragile,allowframebreaks]{\insertsection}{\insertsubsection}
\begin{itemize}
\item In requests possiamo usare l'argomento opzionale \code{params} per
passare un dizionario con le chiavi-valori che vogliamo:
\end{itemize}
{\tiny
\begin{verbatim}
>>> import requests
>>> response = requests.get("http://httpbin.org/get", params={"chiave": "valore"})
>>> print(response.text)
{
  "args": {
    "chiave": "valore"
  },
  ...
  "url": "http://httpbin.org/get?chiave=valore"
}
\end{verbatim}
}
\begin{itemize}
\item In requests l'URL encoding avviene automaticamente
\end{itemize}
\end{frame}

\subsection*{curl: richiesta HTTP con header HTTP personalizzato}
\begin{frame}[fragile,allowframebreaks]{\insertsection}{\insertsubsection}
\begin{itemize}
\item Possiamo passare degli header HTTP personalizzati per avere una
rappresentazione diversa della risorsa o autenticarci con delle credenziali
al server HTTP
\item Possiamo fare ciò con l'opzione \verb|--header|
\end{itemize}
{\tiny
\begin{verbatim}
$ curl --header 'X-Custom: value' http://httpbin.org/get
{
  "args": {},
  "headers": {
    ...,
    "X-Custom": "value"
  },
  ...
  "url": "http://httpbin.org/get"
}
\end{verbatim}
}
\end{frame}

\subsection*{requests: richiesta HTTP con header HTTP personalizzato}
\begin{frame}[fragile,allowframebreaks]{\insertsection}{\insertsubsection}
\begin{itemize}
\item In requests, per passare degli header HTTP personalizzati possiamo usare
l'argomento opzionale \code{headers} che accetta un dizionario con HTTP header e
corrispondenti valori:
\end{itemize}
{\tiny
\begin{verbatim}
>>> import requests
>>> response = requests.get("http://httpbin.org/get", headers={"X-Custom": "value"})
>>> print(response.text)
{
  "args": {},
  "headers": {
    ...,
    "X-Custom": "value"
  },
  ...,
  "url": "http://httpbin.org/get"
}
\end{verbatim}
}
\end{frame}

\section{Conclusioni}
\begin{frame}{\insertsection}
\begin{itemize}
\item Abbiamo visto come fare richieste HTTP con \code{curl} e \code{requests}
\item Abbiamo anche dato un'occhiata come avviene lo scambio di informazioni
utilizzando JSON
\item Mettendo insieme ciò possiamo ora "scriptare" le nostre richieste HTTP!
\end{itemize}
\end{frame}

% FIXME: Use bibtex for that!
\section{Riferimenti}
\begin{frame}{\insertsection}
\begin{itemize}
\item \href{https://training.olicyber.it/training}{Materiale Didattico del Portale di allenamento delle Olimpiadi Italiane di Cybersicurezza} 
\item \href{https://developer.mozilla.org/en-US/}{MDN Web Docs} 
\item \href{https://curl.se}{curl} 
\item \href{https://requests.readthedocs.io/}{Requests: HTTP for Humans} 
\end{itemize}
\end{frame}

\end{document}
